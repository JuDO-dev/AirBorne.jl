\chapter{Introduction}
\label{chapter: Intro}

Buying and selling stock provides investors with the opportunity to grow their money and generate a much larger return on investment compared to a regular savings account. The stock market is innately unpredictable and random thus introducing the risk that the investor makes a loss on their initial investment. The investor is uniquely challenged with the problem of maximizing profit whilst simultaneously minimizing risk. For the beginner or average investor, entering the stock market can be a daunting experience. Making the right decisions at the right time is an important factor in determining the difference between success and failure.  

\noindent Using historical data in order to predict future market behaviour is one way attempting to gain an upper hand \cite{tech_analysis}. There are a variety of different indicators used in effort to identify trends in the market or stock at hand. These indicators are generally used on metrics such as price and volume. They range from simple moving averages to more complex indicators \cite{tech_indic}. Additionally, there are more sophisticated machine learning algorithms used to detect and analyze patterns within a stock's price or volume data to predict future price data.

\noindent This project will specifically explore and introduce a new method of analysis to increase the odds of being successful when trading in the stock market. Behavioural Control Theory will be utilised to form a non-parametric representation of historical stock data and give a prediction of future prices. The relevant background will be introduced in Chapter \ref{Chapter: Background}.

\section{Aims \& Objectives}

\subsection {Aims}

\noindent The main aim of this project is to facilitate fast, accurate and high-performance financial trading by implementation of a programming package named \textbf{AirBorne.jl}. The programming language of choice was Julia. Julia was chosen due to its speed in handling financial simulation, high performance as well as ease of implementation and scripting. The package aims to provide the user with a variety of solutions for
analysing and predicting financial securities. A variety of different prediction algorithms will be explored ranging from simple prediction techniques to a more sophisticated behavioural control approach to stock prediction. All techniques used will be compared and contrasted in terms of their feasibility and accuracy for any financial analysis and trading. 

\subsection {Objectives}

The main objectives of the project can be summarized as follows:

\begin{itemize}
    \item To explore a variety of simple prediction techniques for financial stock analysis/prediction
    \item To compare simple prediction methods with  machine learning approaches.
    \item To explore a behavioural control approach to stock prediction/analysis and trading.
    \item To implement an algorithm to buy/sell or hold stock based on the predictions made by the behavioural control method.
    \item To test, evaluate compare and contrast the generated \textbf{profit}, robustness, feasibility and limitations between the simple prediction techniques, machine learning and behavioural control approach.
\end{itemize}

\section{Overall Report Structure}

The complete structure of the report and a summary of all relevant sections can be viewed as follows:

 \begin{itemize}
     \item Introduction \ref{chapter: Intro}
     
     Shown above, the introduction will clearly specify the problem faced and what the aims of the project are. It depicts how the project intends to address the issue at hand and what is to be further discussed in to remainder of the report.
     
     \item Background \ref{Chapter: Background}
     
     Provides all relevant material essential to ensure the reader's full understanding of the report.
     
     \item Requirements Capture \ref{chapter: Requirements} 
     
     Explores the desirable features as well as necessary requirements of the package development in question.
     
     \item Analysis \& Design \ref{chapter: Analysis} 
     
     Elaborates on the Design of package as well as the reasoning behind the creation of certain files and methods.
     
     \item Implementation \ref{chapter: Implement} 
     
     Contains in-depth description of the implementation of the \textbf{AirBorne.jl} package as well as further information on the behavioural control theory implementation. 
     
     \item Testing \ref{chapter: Testing} 
     
     Covers testing plan devised in order the the performance of the implemented solutions as well as establish functional correctness of code.
     
     \item Results \ref{chapter: Results} 
     
     Contains the Results of the performance analysis of the implemented algorithms.
     
     \item Evaluation \ref{chapter: Evaluation} 
     
     A critical analysis of the results, key takeaways and improvements to be made.
     
     \item Conclusion \ref{chapter: Conclusion} 
     
     Summarises the achievements of this thesis in addition to possible future work to be considered.
     
     \item User Guide \ref{chapter: User_Guide}
     
     Information on how to access the implemented package.
     
 \end{itemize}