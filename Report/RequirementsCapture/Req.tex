\chapter{Requirements Capture}
\label{chapter: Requirements}

% \noindent Projects with a deliverable that serves a specific
% function often have an initial phase in which expected
% use is investigated and a brief more detailed than the
% specification is constructed. This would include what
% is necessary, what is desirable, etc in the final
% deliverable. The results of requirements capture
% determine project objectives and are used to inform
% project evaluation.  

% \noindent Requirements capture is important in all projects with
% real-world deliverables, and is often a significant
% amount of work in software projects. Where
% requirements capture is less relevant (for example in
% an analytical ‘research-style’ project) this may be replaced by a detailed description of the project aims
% and objectives in the Introduction or the Background
% sections.

 As the project entails the implementation of a package to be used for  algorithmic trading, several requirements must be met that facilitate the common steps involved in performing the financial analysis. The requirements highlighted include both necessary requirements that are essential in delivering the intended function of the software as well as desirable requirements that are an added feature, serving to enhance the user experience.

\section{Necessary Requirements}

\subsection{Data Manipulation}
As raw data is imported, it needs to be pre-processed such that only relevant and desirable information is used as an input. Therefore, functions need to be provided that facilitate various methods of data manipulation. Data manipulation includes:

\begin{itemize}
    \item Retrieval
    \item Standardisation
    \item Splitting into Train and Test Data
\end{itemize}

\subsection{Data Plotting}
Plotting provides the user with an intuitive visual representation of the numerical data. It is essential for double-checking prediction results against test data as well as recording errors. In financial analysis, plots are the main form of data representation. 

\subsection{Error Analysis}
 The predictions generated from the chosen algorithm are estimates and thus, it is essential to quantify their accuracy. Functions are needed to facilitate effective numerical comparison between results.

\subsection{Simple Prediction Algorithms}
Implementation of several simple prediction algorithms form the basis of performance comparison between prediction techniques. 

\subsection{Behavioural Control Theory Algorithm}
As one of the main methods of prediction explored take a perspective centered around behavioural control theory, methods and functions need to be implemented that perform the necessary steps in forming as well as solving the optimisation problem.

\subsection{Buy/Sell Algorithm}
The most important aspect of the prediction algorithm used is its profit generating potential. The theoretical profit generated by each prediction technique is the final method of comparison in their evaluation. A method needs to be implemented that measures the profit generated over a selected time-period.

\subsection{Performance Investigation}
Besides implementing the algorithms in Julia code, their performance needs to be investigated and quantified, especially the Behavioural Control Theory algorithm in order to gauge its suitability in a financial setting.

\section{Desirable Requirements}
\label{section: desirable_req}

\subsection{Real-Time Recommendations} 
Real-time recommendations of whether to buy, sell or hold a current stock holding would be a desirable addition to the package, providing a live and more interactive experience of analysis.

\subsection{Integration of Machine learning techniques} 
Addition of the machine learning techniques used in last-year's project would give users a third option for financial analysis and prediction.

\subsection {Stock Allocation Recommendation}
Recommended stock allocation would be a welcome addition to the package that would suggest an allocation of stocks based on the evaluation of past market performance using the prediction algorithm of choice, this would mean that several stocks or an entire market would have to be analyzed. 





