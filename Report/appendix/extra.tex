\subsubsection{Momentum}

\noindent The momentum of a stock can be defined as the rate of change of price, meaning how fast or how quickly a stock changes price over a specified time period \cite{??}. \\

\noindent The Relative Strength Index (RSI) is a popular momentum indicator that measures the magnitude of recent price changes in order to evaluate whether the stock is overbought or oversold. Thus, the \textit{strength} of the stock can be thought of as a measure of whether or not it is a good time to enter or exit the market \cite{??}. \\

\noindent The RSI is calculated in a two-step procedure. In general the first step is calculated using already available data of a specified time period\cite{??}.


\begin{equation}
    RSI_1 = 100- \frac{100}{1+ \frac{G_{t}}{L_{t}}}
\end{equation}

\noindent where, \\
G = Average Gain over the specified time-period, t \\
L = Average Loss over the Specified time-period, t \\

\noindent The specified time-period is commonly taken to be $t=14$ days \cite{??}.

\noindent The second step can then be calculated as follows,

\begin{equation}
    RSI_2 = 100 - \frac{100}{1+ \frac{G*(t-1) + C_g}{L*(t-1) + C_L}}
\end{equation}

\noindent where, \\
G = Previous Average Gain  \\
L = Previous Average Loss \\
$C_g$ = Current Gain \\
$C_L$ = Current Loss \\

\noindent The interpretation of RSI values can be found in Appendix \ref{Appendix: RSI_vals}


\subsubsection{ADX}

The direction of a stock can be thought of as an upward or downward price change. The Average Directional Index quantifies the strength of these up/downtrends \cite{??}. The ADX is calculated via a series of different calculations, The Directional Movement (DM) and True Range (TR) needs to be computed for each day of the specified time period. 

\begin{equation}
    +DM_t = CH - PH \\
\end{equation}
\begin{equation}
    -DM_t = CL - PL \\
\end{equation}
    


\noindent where, \\
$+DM_t$ = Positive Directional Movement at time, t \\
$-DM_t$ = Negative Directional Movement at time, t \\
CH = Current High \\
PH = Previous High \\
CL = Current Low \\
PL = Previous Low \\

\noindent We then assign each day, a $+DM$ or $-DM$ and TR. $+DM$ is used if $+DM$ > $-DM$ or vice versa. \\

\noindent True Range is defined to be the greater of CH - CL, CH - PC, CL-PC, where PC is previous closing price. The Average True Range (ATR) is then calculated as follows, \\

\begin{equation}
    ATR = \sum^{t}_{t=1}TR_t - \frac{\sum^{t}_{t=1}TR_t}{t} + CTR
\end{equation}

\noindent where, \\
CTR = Current True Range

\noindent The smoothed $+/-DM$ is then calulated for each day and assigned.

\begin{equation}
    +/-DM_{smooth} = \sum^{t}_{t=1}DM_t - \frac{\sum^{t}_{t=1}DM_t}{t} + CDM
\end{equation}

\noindent where, 
CDM = Current Directional Movement

\noindent The Directional Indicator (DI) and in turn Directional Index (DX) the is then calculated for each day.

\begin{equation}
    +/-DI = \frac{+/-DM_{smooth}}{ATR}* 100\%
\end{equation}

\begin{equation}
    DX_t = \frac{|(+DI_t)-(-DI_t)|}{|(+DI_t)+(-DI_t)|}
\end{equation}
where, \\
$DX_t$ = Direction index at time, t \\
$+DI_t$ = Positive Directional indicator at time, t \\
$-DI_t$ = Negative directional indicator at time, t \\

The first $ADX_1$ and every ADX after ($ADX_2$) is given to be,
\begin{equation}
    ADX_1 = \frac{\sum^{t}_{t=1}DX_t}{t}
\end{equation}
\begin{equation}
    ADX_2 = \frac{ADX_p * (t-1) + CADX}{t}
\end{equation}

\noindent where, \\
$ADX_p$ = Prior ADX
CADX = Current ADX
t = specicfied time-period

\noindent In general, a "good" time-period is $t=14$ days \cite{??}.
